% =========================================================
%  Preamble común — paquetes y configuración compartida
%  (Úsalo desde libros y plantillas Cornell)
% =========================================================

% ------------------------------
% Codificación, tipografías, idioma
% ------------------------------
\usepackage[utf8]{inputenc}   % Entrada en UTF-8 (acentos directos)
\usepackage[T1]{fontenc}      % Codificación de salida T1 (mejor separación silábica)
\usepackage{lmodern}          % Familia Latin Modern (mejor que Computer Modern)
% \usepackage[spanish]{babel} % (OPCIONAL) reglas tipográficas/es de español

% ------------------------------
% Maquetación y gráficos
% ------------------------------
\usepackage{geometry}         % Márgenes y tamaño de página
\geometry{margin=2.5cm}       % Margen por defecto del monorepo
\usepackage{graphicx}         % Inserción de figuras/imágenes
\usepackage{xcolor}           % Colores (para hipervínculos, cajas, etc.)

% ------------------------------
% Hipervínculos
% ------------------------------
\usepackage{hyperref}         % Enlaces internos/externos
\hypersetup{
    colorlinks=true,
    linkcolor=blue,           % Color de enlaces internos (TOC, refs)
    urlcolor=blue,            % Color de URLs
    citecolor=blue            % Color de citas (biblatex)
}

% ------------------------------
% Estructura de párrafos y matemáticas
% ------------------------------
\usepackage{parskip}          % Párrafos sin sangría + espacio vertical
\usepackage{amsmath, amssymb} % Entornos y símbolos matemáticos

% ------------------------------
% URLs largas y microtipografía (mejora de corte de línea)
% ------------------------------
\usepackage{xurl}                 % Permite cortes de línea dentro de \url
\PassOptionsToPackage{hyphens}{url}{} % Mejora cortes con hyperref/url
\urlstyle{same}                   % Misma fuente para \url que el texto
\usepackage{microtype}            % Ajustes sutiles de espaciado (mejor legibilidad)
\emergencystretch=3em             % (Opcional) Relaja el corte de línea en párrafos problemáticos

% ------------------------------
% Bibliografía con biblatex/biber
% ------------------------------
\usepackage{csquotes} % Recomendado por biblatex (gestión de comillas)
\usepackage[backend=biber,style=authoryear,sorting=nyt]{biblatex}

% Carga ADAPTABLE del .bib según el modo de compilación:
% - Cuando compilas con latexmk usando aux/out en latex.out (build del monorepo)
% - Cuando compilas "in-place" (make refs) dejando main.pdf junto al .tex
% - Como último recurso, cuando compilas desde la raíz del repo
\makeatletter
\IfFileExists{../../../../common/bibliography.bib}{%
  % Caso 1: compilación con latex.out (el .aux vive en .../book/latex.out/)
  \addbibresource{../../../../common/bibliography.bib}%
}{%
  \IfFileExists{../../../common/bibliography.bib}{%
    % Caso 2: compilación in-place (el .aux vive en .../book/)
    \addbibresource{../../../common/bibliography.bib}%
  }{%
    \IfFileExists{common/bibliography.bib}{%
      % Caso 3: compilación desde la raíz del repositorio
      \addbibresource{common/bibliography.bib}%
    }{%
      % Aviso para depuración si no se encuentra el .bib
      \typeout{[WARN] bibliography.bib no encontrado en rutas conocidas.}%
    }%
  }%
}
\makeatother
