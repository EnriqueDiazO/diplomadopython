\chapter{Introducción}
Python es un lenguaje de programación de alto nivel, interpretado y de propósito general, conocido por su sintaxis legible y concisa.
Fue creado a finales de los años 80 por Guido van Rossum en el centro de investigación CWI (Países Bajos) como sucesor del 
\href{https://docs.python.org/3/license.html}{lenguaje ABC}. 

El nombre “Python” no proviene del reptil, sino del grupo de comedia británico Monty Python – Van Rossum buscaba un nombre corto, 
único y misterioso, inspirándose en los guiones de Monty \href{https://docs.python.org/3/faq/general.html}{\textit{Python’s Flying Circus}} 
que leía durante el desarrollo. Python fue liberado por primera vez en 1991 y desde entonces ha evolucionado con contribuciones de una amplia 
comunidad, manteniendo a Guido van Rossum como su \textit{“Benevolent Dictator For Life”} hasta su retiro del cargo en 2018.

\section{Lenguaje de Programación Python}
Desde sus inicios, Python ha destacado por su filosofía de diseño enfocada en la legibilidad y la productividad del desarrollador. 
El famoso \href{https://peps.python.org/pep-0020/}{“Zen de Python”} (PEP 20 o \href{https://peps.python.org/pep-0000/}{\textit{Python Enhancement Proposals}}) resume principios como “la legibilidad cuenta” y “lo simple es mejor que lo complejo”. 
En la práctica, Python se utiliza en una gran variedad de áreas de aplicación: desarrollo web, automatización de tareas administrativas, 
análisis de datos, inteligencia artificial, ciencia de datos, scripting, computación científica, desarrollo de videojuegos, entre otros.

\section{Áreas de Aplicación de Python}
Su amplia biblioteca estándar y el ecosistema de paquetes externos (a través de \href{https://pypi.org/}{PyPI}) ofrecen herramientas para 
casi cualquier problema, desde el procesamiento de texto y operaciones matemáticas hasta protocolos de Internet y aprendizaje automático. 
Gracias a su filosofía multiparadigma (soporta programación estructurada, orientada a objetos y funcional) y a su sintaxis clara, 
Python se utiliza en innumerables ámbitos. En desarrollo web, marcos como \href{https://www.djangoproject.com/}{Django} o \href{https://flask.palletsprojects.com/en/stable/}{Flask} 
permiten crear aplicaciones robustas. En ciencia de datos y aprendizaje automático, bibliotecas como \href{https://numpy.org/}{NumPy}, 
\href{https://pandas.pydata.org/}{pandas}, \href{https://scikit-learn.org/stable/}{scikit-learn} o \href{https://www.tensorflow.org/learn?hl=es-419}{TensorFlow}
son estándar de la industria para análisis estadístico, manipulación de datos y redes neuronales. Python también es muy usado en 
automatización de tareas de sistemas, \href{https://realpython.com/tutorials/devops/}{devOps} y \href{https://www.redhat.com/en/blog/python-scripting-intro}{scripting}, 
mediante módulos como os, sys o subprocess que facilitan escribir utilidades para  administración de servidores o integración continua. 
En computación científica, herramientas como \href{https://scipy.org/}{SciPy} y \href{https://matplotlib.org/}{Matplotlib} ofrecen capacidades de cómputo numérico y visualización, convirtiendo 
a Python en un “laboratorio virtual” para físicos e ingenieros. Incluso en desarrollo de videojuegos (v.g. con \href{https://www.pygame.org/news}{Pygame} 
o \href{https://godotengine.org/asset-library/asset/3234}{Godot Engine}), aplicaciones de escritorio (\href{https://doc.qt.io/qtforpython-6/}{PyQt}, \href{https://docs.python.org/3/library/tkinter.html}{Tkinter}) 
o computación en la nube (SDKs de AWS, Google Cloud), Python provee soluciones. 
Esta versatilidad, sumada a una comunidad activa que produce abundante documentación y paquetes, ha consolidado a Python como uno de los 
lenguajes más populares y enseñados en la actualidad.


\begin{lstlisting}[language=Python, caption={El Zen de Python}]
>>> import this
  "Beautiful is better than ugly".
  "Explicit is better than implicit."
  "Simple is better than complex."
  "Complex is better than complicated."
  "Flat is better than nested."
  "Sparse is better than dense."
  "Readability counts."
  "Special cases aren't special enough to break the rules."
  "Although practicality beats purity."
  "Errors should never pass silently."
  "Unless explicitly silenced."
  "In the face of ambiguity, refuse the temptation to guess."
  "There should be one-- and preferably only one --obvious way to do it."
  "Although that way may not be obvious at first unless you're Dutch."
  "Now is better than never."
  "Although never is often better than *right* now."
  "If the implementation is hard to explain, it's a bad idea."
  "If the implementation is easy to explain, it may be a good idea."
  "Namespaces are one honking great idea -- let's do more of those!"
\end{lstlisting}


\section{Python 2 vs Python 3}

Un hito importante en la evolución del lenguaje fue la transición de Python 2 a Python 3. Python 2.x fue la serie principal durante 
muchos años, pero presentaba ciertas limitaciones y decisiones de diseño heredadas que el equipo de desarrollo decidió corregir aun 
a costa de romper compatibilidad hacia atrás. En 2008 se lanzó Python 3.0 (“Py3k”), una versión intencionalmente incompatible con Python 2.x, 
que resolvió muchos de esos \href{https://peps.python.org/pep-0404/#:~:text=Python%20is%20,there%E2%80%99s%20also}{“wart”} del lenguaje.
Por ejemplo:

\begin{itemize}
    
    \item En \textit{Python 3} la función \textcolor{blue}{print()} sustituyó a la sintaxis de \href{https://docs.python.org/3/whatsnew/3.0.html}{\textit{statement \textcolor{blue}{print}}} de 
\textit{Python 2}.

 \item  Las cadenas de texto son Unicode por defecto (en \href{https://peps.python.org/pep-0404/#:~:text=Strings%20and%20bytes}{Python 2 existían tipos separados para bytes y texto}, lo que ocasionaba confusiones).
 
 \item La división de enteros mediante \/ produce un número de punto flotante (antes en Python 2 la operación \/ entre enteros truncaba el resultado a entero.

 \item En Python 3 existe solo el tipo int de precisión ilimitada, eliminando el antiguo \href{https://peps.python.org/pep-0404/#:~:text=Python%202%20has%20two%20basic,type}{\textit{long}}
\end{itemize}

Python 3 también simplificó las reglas de comparación de objetos y eliminó métodos obsoletos. Tras un largo periodo de coexistencia, 
Python 2 alcanzó su fin de vida en 2020 (su última versión fue 2.7) y desde entonces Python 3 es la versión recomendada y mantenida 
activamente. Hoy prácticamente todas las bibliotecas populares soportan solo Python 3, aprovechando sus mejoras de rendimiento y 
características modernas.



\section{Instalación de Python y Configuración del Entorno}
\href{https://packaging.python.org/en/latest/tutorials/installing-packages/}{Instalar Python} es el primer paso para empezar a desarrollar. Python es software libre y multiplataforma: se puede obtener gratuitamente desde 
el \href{https://www.python.org/downloads/}{sitio oficial de Python} para \href{https://www.python.org/downloads/windows/}{Windows}, \href{https://www.python.org/downloads/macos/}{macOS}
 y distribuciones \href{https://www.python.org/downloads/source/}{Linux}

\subsection{Windows}
En sistemas Windows, el instalador oficial agrega opcionalmente Python al PATH del sistema y provee el lanzador py.exe para facilitar la invocación 
de diferentes versiones. 

\subsection{Linux y Mac}

En macOS y Linux, a menudo ya existe alguna versión de Python instalada (particularmente Python 2 en versiones antiguas de sistemas Unix), 
por lo que suele ser recomendable instalar la versión más reciente de Python 3 desde los paquetes oficiales o el gestor de paquetes del 
sistema operativo. Para verificar la instalación, se puede ejecutar python3 --version en la terminal y comprobar que imprime un número 
de versión. A partir de Python 3.4, el instalador oficial incluye la herramienta de gestión de paquetes pip y en Windows el lanzador mencionado,
lo que simplifica la configuración inicial.

Una vez instalado, es importante contar con un entorno de desarrollo cómodo. \href{https://code.visualstudio.com/}{Visual Studio Code (VS Code)}, 
un editor de código multiplataforma de Microsoft, se ha vuelto muy popular para programar en Python. VS Code ofrece una extensión oficial de Python 
que proporciona realce de sintaxis, autocompletado \textit{IntelliSense}, depuración interactiva, análisis estático de código (\textit{linting}) 
y gestión de entornos virtuales, entre otras \href{https://code.visualstudio.com/docs/languages/python#:~:text=Working%20with%20Python%20in%20Visual,including%20virtual%20and%20conda%20environments}{características}.
Con VS Code, por ejemplo, es posible ejecutar o depurar un script Python con solo pulsar un botón, visualizar variables en tiempo real y saltar entre
 funciones paso a paso, lo que agiliza enormemente el ciclo de desarrollo. Además, la integración con control de versiones Git y otras extensiones 
 (p.ej., para Docker o notebooks de Jupyter) hacen de VS Code un entorno muy versátil.

Otro componente fundamental en el ecosistema de Python, especialmente en ciencia de datos, son los \href{https://code.visualstudio.com/docs/datascience/jupyter-notebooks#:~:text=Jupyter%20Notebooks%20in%20VS%20Code,on%20one%20canvas%20called}
{Jupyter Notebooks}. Jupyter Notebook es una aplicación web interactiva que permite combinar en un mismo documento texto explicativo, 
código ejecutable, resultados computacionales y visualizaciones gráficas. Los notebooks facilitan un estilo de programación exploratorio: 
el usuario puede escribir fragmentos de código (celdas) en Python y ejecutarlos inmediatamente para ver sus resultados (por ejemplo, tablas de datos 
o gráficos), junto con descripciones, fórmulas matemáticas y anotaciones. 
Esto es ideal para análisis de datos, experimentos científicos, o demostraciones educativas, ya que documenta el proceso junto con el código. 
Para usar Jupyter, se puede instalar el paquete jupyter (o JupyterLab, una versión más reciente) con pip, y luego lanzar el servidor con jupyter 
notebook o jupyter lab, que abrirá una interfaz web local. Dentro de VS Code también existe integración para notebooks: la extensión de Python permite 
abrir archivos .ipynb con una interfaz interactiva muy similar a Jupyter, ejecutando celdas de código y mostrando sus salidas, todo dentro del editor

En este diplomado de ciencia de datos, es común que los ejercicios y ejemplos se realicen en notebooks, por la comodidad de ir alternando explicación 
y ejecución de código.


Por lo tanto, el proceso típico de preparación del entorno de Python incluirá: instalar Python 3 desde la fuente adecuada; opcionalmente instalar un entorno de desarrollo (como VS Code con la extensión de Python); y familiarizarse con el uso de notebooks de Jupyter si se trabajará en análisis interactivo. Con estas herramientas configuradas, estaremos listos para escribir y ejecutar programas Python eficientemente.
\begin{table}[ht]
    \centering
    \begin{tabular}{|c|c|}
        \hline
        Column A & Column B \\
        \hline
        1 & 2 \\
        3 & 4 \\
        \hline
    \end{tabular}
    \caption{This is an example table.}
    \label{tab:example}
\end{table}